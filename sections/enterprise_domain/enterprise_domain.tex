\chapter{Enterprise Domain}

"Patterns of Enterprise Application Architectures" - "November 05, 2002"

"Enterprise applications often have complex data—and lots of it—to work on, together with business rules that fail all tests of logical reasoning. Although some techniques and patterns are relevant for all kinds of software, many are relevant for only one particular branch."

"Usually many people access data concurrently."

"there are still problems in making sure that two people don't access the same data at the same time in a way that causes errors."

"built at different times with different technologies"

"If you can do things that improve small projects, then that cumulative effect can be very significant on an enterprise, particularly since small projects often have disproportionate value. Indeed, one of the best things you can do is turn a large project into a small one by simplifying its architecture and process."

"Scalability is a measure of how adding resources (usually hardware) affects performance. A scalable system is one that allows you to add hardware and get a commensurate performance improvement, such as doubling how many servers you have to double your throughput. Vertical scalability, or scaling up, means adding more power to a single server, such as more memory. Horizontal scalability, or scaling out, means adding more servers."