\chapter{Interview with Danske Bank}
\label{app:interview}

In corporation with the Danske Bank and their \textit{Corporate Users and Agreements} department a suitable legacy system in their infrastructure was identified for analysis. The following is  some of the key statements from the Lead Software Architect Thomas Lønborg Hansen in the follow up meeting about the BCA system on May the 5th, 2017. Each statement is either related to a question about the system or a topic.

\textbf{General comments on the nature of the system}\\
"... the first version of the system was developed in 1991 .. in the beginning [of the development process], that was what you did at the time"\\

"the database itself is giant monolith setup where a lot of applications are utilizing the same physical database .. a classic mainframe design that creates huge benefits for availability for each other [among applications] and often gives high speeds because the database connection can be reused throughout the application .. at the same time this incurrs a tightly coupling amongst applications"\\

\textbf{How many applications are dependent on the BCA system?}\\
"There are 20 or 30 system each consisting of 50 or more applications, they mainly query the database through interfaces, while some legacy systems still directly interact with the database"\\

\textbf{What do you mean when you say there are a proxy in front of the database?}\\
"there are several kinds of interfaces [to the central database], some are pure mainframe interfaces, others are more API similar in different variants"\\
"For a long time smaller parts of the legacy system was integrated [into the central database]"\\

\textbf{Challenges with a shared database}\\
"many of the operations will have similar logic, because several teams [the different applications] need the same logic, because it is the same they need"\\
"before the first iteration, every application took outset in the database tables ... some applications had certain logic, directly operating on the database .. others applications only read from the database, implementing their own SQL statements to read from the database"\\
"this [the central database] was a classic issue with legacy systems that was apparent because you [generally as a software developer] had a different approach to developing applications back then [in 1991 when the BCA was originally created] .. the database was the integration layer"\\
"A lot of people needed the same functionality, which caused a lot of duplicated code, maybe to even 100 callers [having the same needs]"\\


\textbf{Read to write ratio}\\
"read to write ratio is really high .. maybe 10.000 or even 100.000 read to write ratio .. this is very standard for a administration system"\\

\textbf{Which size is the database?}\\
"the database contains approximate a 100 tables .. with a event table for each, meaning that there are something along the lines of 200 tables total"\\

\textbf{What kind of information is stored in the database?}\\
"besides access control on user access to data you also have access control on functionality .."\\