\chapter{Evaluation}

This chapter evaluates the two cases answer to the problem statements.

\section{Consistency}
Client-centric Benchmarking of Eventual Consistency for Cloud Storage Systems: "client-centric consistency, which captures what client applications observe directly, or system-centric consistency, which captures the convergence of the storage system’s replication protocol"

'Toward a Principled Framework for Benchmarking Consistency'
He total order extends the “happens before” partial order (i.e., if operation A ended before op- eration B began during the execution, then A precedes B in the total order);

\section{Availability}
'Benchmarking Cloud Serving Systems with YCSB'

"Replication is used to improve system availability (by directing traffic to a replica after a failure), avoid data loss (by recovering lost data from a replica), and improve performance (by spreading load across multiple replicas and by making low-latency access available to users around the world)."

\subsection{Measurement points}
How does the amount of data stored affect the performance?
How is consistency managed?

\subsection*{Notes}

The STREAM benchmark is a simple synthetic benchmark program that measures sustainable memory bandwidth (in MB/s) and the corresponding computation rate for simple vector kernels. 

\url{http://www.cs.virginia.edu/stream/ref.html}


'Benchmarking Cloud Serving Systems with YCSB' : 'Classification of Systems and Tradeoffs'
Read performance versus write performance, Latency versus durability, Synchronous versus asynchronous replication, Data partitioning