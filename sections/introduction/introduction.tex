\chapter{Introduction}
\label{ch:introduction}
\comment{Skal tilpases ny problemformulering}
The requirements for enterprise application availability, performance and maintainability is constantly rising. Stemming from high amounts of information with unpredictable and varying load, together with a demand for failure free and feature rich platforms. These requirements has opened a new market for big scale cloud-computing services, providing server infrastructure and tools which help resolve these new challenges. A lot of new possibilities has opened with many open source initiatives, giving new possibilities when deploying, scaling and updating applications, with containerization as the enabling technology. The cloud-computing services are constantly expanding and reiterating tools, trying to capture the market.



\section{Problem formulation}
\label{sc:problem_formulation}

\begin{itemize}  
\item 1.a How can cloud computing be used to solve availability, performance and maintainability requirements for enterprise applications?
\item 1.b How can cloud computing be used in an enterprise application architecture?

\item 2.a How are enterprise application challenges and possible solutions affected by the application domain?
\item 2.b How does the enterprise application domain affect challenges and possible solutions?
\item 2.c How does the domain shape the challenges in a specific application architecture?
\item 2.d How does the domain and organization shape the challenges in a specific application architecture?

\item 3.a How does a distributed application architecture deal with a single point of failure?
\item 3.b How is single point of failure avoided in a distributed application architecture?

\item 3.c How is the optimal application architecture and distribution method determined and evaluated?
\item 3.d How is the enterprise application architecture determined?
\item 3.e How is the enterprise application architecture evaluated?
\end{itemize}

\subsection*{Notes}
How are the challenges and their optimal solution affected by the specific application?

What determines the optimal application architecture, database and distribution?

How is the optimal application architecture, database and distribution determined?

System architecture:
How is application data optimally stored?
How is application data optimally distributed?

How are application updates optimally executed?
How are updates 

How is the system distributed

This area has been named cloud computing, 

Big cloud computing platforms are now available, offering server rental with 

From these requirements big cloud platforms have been started, 

These requirements has started big scale initiatives in cloud computing, has generated new tools and databases available for development of distributed big scale enterprise applications. 

Creating a need for developers to explore and evaluate which of the many platforms to choose, 

\note{

Kig i kapitel 2 s. 59 - cloud-native-java-designing-resilient-systems-with-spring-boot-spring-cloud-and-cloud-foundry
"The patterns for how we develop software, both in teams and as individuals, are always evolving. The open source software movement has provided the software industry with somewhat of a Cambrian explosion of tools, frameworks, platforms, and operating systems—all with an increasing focus on flexibility and automation. A majority of today’s most popular open source tools focus on features that give soft‐ ware teams the ability to continuously deliver software faster than ever before possi‐ ble, at every level, from development to operations."

%\subsection*{Noter}
These requirements has started big scale initiatives in cloud computing, has generated new tools and databases available for development of distributed big scale enterprise applications. 

Creating a need for developers to explore and evaluate which of the many platforms to choose, 

has generated new tools and databases available for development of distributed big scale enterprise applications.

platform for automating deployment, scaling, and operations of application containers across clusters of hosts

Which level are we on? High level: Google app engine, or low level abstraction: EC2.

"The Antifragile Organization" - \url{http://queue.acm.org/detail.cfm?id=2499552}
"Failure is inevitable. Disks fail. Software bugs lie dormant waiting for just the right conditions to bite. People make mistakes. Data centers are built on farms of unreliable commodity hardware. If you're running in a cloud environment, then many of these factors are outside of your control. To compound the problem, failure is not predictable and doesn't occur with uniform probability and frequency. The lack of a uniform frequency increases uncertainty and risk in the system. In the face of such inevitable and unpredictable failure, how can you build a reliable service that provides the high level of availability your users can depend on?"

Why is it worth it to take on this monster of instability?


"Weathering the Unexpected" - \url{http://queue.acm.org/detail.cfm?id=2371516}

"Whether it is a hurricane blowing down power lines, a volcanic-ash cloud grounding all flights for a continent, or a humble rodent gnawing through underground fibers—the unexpected happens. We cannot do much to prevent it, but there is a lot we can do to be prepared for it. To this end, Google runs an annual, company-wide, multi-day Disaster Recovery Testing event—DiRT—the objective of which is to ensure that Google's services and internal business operations continue to run following a disaster."


Disasters happen for all kinds of systems. So lets prepare for it already in implementation.


"Site Reliability Engineering" - (page 17 in downloaded pdf)

"Software engineering has this in common with having children: the labor before the birth is painful and difficult, but the labor a er the birth is where you actually spend most of your effort. Yet software engineering as a discipline spends much more time talking about the first period as opposed to the second, despite estimates that 40–90% of the total costs of a system are incurred after birth.1"


Software systems incur a lot of work even after deployment, so lets spend some time here.



"Toward Antifragile Cloud Computing Infrastructures"\cite{abid2014toward}
"Cloud computing systems are rapidly growing in scale and complexity. They are also changing dynamically as a result of dynamic addition and removal of system components, different execution environments, common updates and upgrades, runtime repairs, mobility of devices and more. Such large-scale, complex and dynamic cloud environments are prone to failures and performance anomalies"


"Resilience and Survivability in Communication Networks: Strategies, Principles, and Survey of Disciplines"\cite{sterbenz2010resilience}
"The Internet has become essential to all aspects of modern life, and thus the consequences of network disruption have become increasingly severe. It is widely recognised that the Internet is not su ciently resilient, survivable, and dependable, and that significant research, development, and engineering is necessary to improve the situation."

"Networks in general, and the Global Internet in particular, have become essential for the routine operation of businesses and to the global economy. Consumers use the Internet to access information, obtain products and services, manage finances, and communicate with one another. Businesses use the Internet to transact commerce with consumers and other businesses."



"Toward Antifragile Cloud Computing Infrastructures" "Amal Abida,b, Mouna Torjmen Khemakhema, Soumaya Marzouka, Maher Ben Jemaaa, Thierry Monteilb,c, Khalil Drirab,c"
"Cloud computing is earning an increasing popularity over traditional information processing systems for storing and processing huge volumes of data. This concept consists in offering services and resources on-demand over the Internet in the pay-as-you-go model. The Cloud infrastructure is built on modern data centers covering thousands of interconnected servers with capability of hosting a large number of applications. These data centers are often virtual- ized and computing resources are provisioned to the user in the form of configurable Virtual Machines (VMs)."



Stability summary chapter 6 \cite[p. 117]{nygard2007release}

"Astronomically unlikely coincidences happen daily"
"Failures are inevitable. Our systems, and those we depend on will fail in ways large and small. Stability antipatterns amplify transient events. They accelerate cracks in the system. Avoiding the antipatterns does not prevent bad things from happening, but they wull help minimize the damage when bad things do occur."


"A View of Cloud Computing"
"While the cost of over provisioning is easily measured, the cost of under provisioning is more difficult to measure yet potentially equally serious: not only do rejected users generate zero revenue, they may never come back"
"In fact, this example underestimates the benefits of elasticity, because in ad- dition to simple diurnal patterns, most services also experience seasonal or other periodic demand variation (for example, e-commerce in December and photo sharing sites after holidays) as well as some unexpected demand bursts due to external events "
}

\note{
“1.4. Attack of the Clusters
At the beginning of the new millennium the technology world was hit by the busting of the 1990s dot-com bubble. While this saw many people questioning the economic future of the Internet, the 2000s did see several large web properties dramatically increase in scale.”
Excerpt From: “NoSQL Distilled: A Brief Guide to the Emerging World of Polyglot Persistence.” 


“One is to handle data access with sizes and performance that demand a cluster; the other is to improve the productivity of application development by using a more convenient data interaction style”
Excerpt From: pramod j. sadalage. “NoSQL Distilled: A Brief Guide to the Emerging World of Polyglot Persistence.” iBooks. 

}