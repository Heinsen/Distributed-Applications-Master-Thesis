\chapter{Software development is changing}
\label{ch:introduction}
Increased competition and complexity of the problems companies are solving today directly impact the entire software development industry. Software development is getting increasingly more focused towards having short and independent release cycles. Cloud computing makes it possible to buy computing resources as they are needed, cheap and fast. New languages and open source frameworks makes it possible to create nice and feature rich applications very quickly. Together with new approaches to creating successful startup businesses, former existing barriers for utilizing new business opportunities have been removed. In this new and competitive environment, understanding and solving complex problems is a process that has to be repeated daily and quickly. Software developers now need to understand the domain and the context they are developing software for in a much higher degree than ever before. Products need to be available to the customer immediately, and at all times, downtime is immediately visible on the bottom line.


\section{Small iterations}
Solving problems is hard, when complexity is high and time to market is crucial, this is Fred George's outset in his 2016 talk at GOTO Stockholm\cite{george2016it}. According to George, the complexity of problems determine the value of solving them. He states that the best solution is found through many small iterations, where features are independently developed through a high amount of iterations. George emphasizes the need for speeding up development iterations showing the decreasing length of iterations from the early 2000 to today. George utilized scrum in the teams he was collaborating with in the early 2000, but shifted to extreme programming, cutting down on iteration length. According to George today's agile development iteration length should be less than one week, personally preferring a iteration length of a single day. Stating that a very shot iteration length improves productivity immensely. By having a short iteration time, it is possible to identify a solution to a complex problem quickly. Further stating that there are three inhibitors for speeding up iteration length: technology, process and organisation. Technology inhibitors include not utilizing cloud computing, having a unified programming language that is used to develop everything, and not using open source frameworks that can help developers maintain delivery speed. At the same time George emphasize that the architecture of the system has to change accordingly as well, a central SQL database slows down the development teams, and many startup companies successfully have move to a central event bus. Process inhibitors include not understanding the problem and the domain around it, and not correctly identifying the final requirements through several iterations. Organisational inhibitors include introducing too many titles, changing team structure and having dedicated leaders.


\section{Knowledge sharing}
A huge trend of knowledge sharing has started, where online accessible conference talks and a high quantity of open source projects inspire and help developers solve common challenges. Conferences have focus on new open source technology, system architecture, working process and organisational structure, all with a common goal: speeding up software development and the ability to solve increasingly difficult problems. Conferences are either entirely committed to or has tracks about topics like: Cloud computing, Agile development, Domain Driven Design, Microservices, NoSQL, Docker, Cassandra among many others. Company internal projects that beforehand was only internal accessible, are increasingly made open source, giving developers very advanced tools to create elegant and efficient solutions quickly. These projects are freely accessible, with well known support challenges, where developers can find help with specific challenges utilizing the projects.

The entire movement started when Amazon announced Amazon Web Services in 2006. It meant anyone could register and rent virtual machines for hosting distributed applications. Since then, virtualization has been further revolutionised with the introduction of container technology, which was made open source through the docker initiative in 2014. Docker made it possible to start up new independent and isolated environments in seconds, something Google had developed on internally because they needed it. Which is somewhat a trademark for the entirety of the cloud native movement, is has sprung out of a big variety of needs: serving massive amounts of users, creating reliable applications, the ability to update and add functionality often and easily, minimize integration complexity and technology lock in. In the end the focus should be on solving the problems at hand and in the future, not the maintenance and extension of the existing application.

Eric Evans talked about domain driven design (DDD) at the \textit{Domain Driven Design Europe} conference\cite{evans2016tackling} in 2016, focusing on the role of DDD in state of the art development. Evans explains how available development technologies have diversified, and developer insight and knowledge is very good. Monolith applications were a result of technological boundaries, and made developing good design a major challenge. According to Evans these boundaries have disappeared, pushing developers towards more thoughtful design of applications. Developers today need to put emphasis on understanding the underlying challenges in the particular domain, identifying the optimal architecture that supports the context.

\newpage
\section{Problem statement}
\label{sc:problem_statement}

\begin{itemize}
\item How can a microservice architecture solve problems with a monolith architectures?
\item What does having a resilient system entail, and how is resilience achieved in a software application context?
\end{itemize}


\comment{Vi vil gerne undersøge dette område, da nuværende systemer er blevet for store og svære at vedligholde, udvide og introducere til andre}
\comment{Vi vil gerne undersøge resilience fordi det er har danske bank sagt de gerne vil opnå, måske en 'smule naivt' uden at vide hvad det præcist betyder}


\newpage
\comment{Gamle formuleringer}
\begin{itemize}  
\item 1.a How can cloud computing be used to solve availability, performance and maintainability requirements for enterprise applications?
\item 1.b How can cloud computing be used in an enterprise application architecture?

\item 2.a How are enterprise application challenges and possible solutions affected by the application domain?
\item 2.b How does the enterprise application domain affect challenges and possible solutions?
\item 2.c How does the domain shape the challenges in a specific application architecture?
\item 2.d How does the domain and organization shape the challenges in a specific application architecture?

\item 3.a How does a distributed application architecture deal with a single point of failure?
\item 3.b How is single point of failure avoided in a distributed application architecture?
\item 3.c How is the optimal application architecture and distribution method determined and evaluated?
\item 3.d How is the enterprise application architecture determined?
\item 3.e How is the enterprise application architecture evaluated?
\item 3.f What determines the optimal application distribution method?
\item 3.g How is the optimal application database determined?
\end{itemize}