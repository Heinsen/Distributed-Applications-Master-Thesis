\chapter{Architecture}
\label{ch:architecture}

\section{Monolith}




\section{Service-Oriented}




\section{Microservices: Theory and Application}
The thought of microservies is not new. But the accessibility and knowledge sharing is.

\url{https://www.youtube.com/watch?v=bHqRxMwfrng}

Monolith:
Typically kind of bad, as they grow and get more and more unmaintainable the cost of maintenance outpases the outweighs the benefits. Implementing a new feature takes a long time.

Mircroservices:
Domain driven design, understand what you are building. Break apart your business functions around bounded context, a bounded context per service.

Principals

\begin{itemize}
\item Encapsulation
\item Automation
\item Domain centric
\item Decentralized
\item Independent
\item Fail-safe
\item Observable
\end{itemize}

Scalability 

Shift to maintainability. Writing fine grained very focused micro services. It is modular, easier readable, understandable.

'Sea change' - Time is ripe. Automation, containers, Dev-Ops, higher level abstraction.

Team structure should usually be a holistic end to end team QA, product management, developers, release engineers, working together front to end. They own the product, service and so on. Ties the team together.

Partitioning strategy Verb or use-case, noun or resource, grouping things that change together. Single responsibility principle.

Benefits. Faster deployment. Easier to test. Scalability.

Challenges.
More complexity. System testing. Distributed transactions, (eventual consistency). Management of system.
Organization and culture, maturity.

Netflix
\url{https://www.youtube.com/watch?v=57UK46qfBLY}
Microservice failure
\begin{itemize}
\item Hystrix
\item Chaos monkey
\item Fault-injection test framework
\end{itemize}